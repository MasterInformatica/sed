\section{Conclusiones}
\label{s4:sec:Conclusiones}

En lo expuesto en esta memoria y el trabajo realizado creemos que se
ha realizado con éxito el juego \pong utilizando maletines ARM y una
FPGA comunicados mediante UARTs y visualizado en un monitor VGA. \\

El trabajo realizado supera nuestras espectativas iniciales de
conectar sólo dos maletines al añadir la FPGA y un monitor VGA. \\

Somos conscientes de algunas deficiencias en el sistema de
comunicación que, aún siendo funcional, podrían mejorar. Y lo dejamos
para un posible trabajo futuro junto con la idea de añadir más
jugadores. \\

Esta última idea no debería ser complicada, ya que el sistema de
propagación de comandos y puntuaciones en los maletines no tiene
número máximo establecido, las limitaciones se nos plantean en la
lógica del juego que está pensado para máximo dos jugadores, pero
sustituyendo los muros superior e inferior por raquetas de los nuevos
jugadores, creemos viable realizar la ampliación a cuatro jugadores. \\

Queremos por último mostrar nuestra satisfacción con el trabajo
realizado ya que consideramos haber conseguido integrar en un proyecto
lo aprendido en diferentes asignaturas tanto del máster como del
grado. 
%
%
%%%
%%% Local Variables:
%%% mode: latex
%%% TeX-master: "../main.tex"
%%% End:


