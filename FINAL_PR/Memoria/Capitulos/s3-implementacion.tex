\section{Implementación}
\label{s3:sec:Implementacion}


\subsection{FPGA}
\label{s3:subsec:fpga}
Para la implementación del juego sobre la FPGA, se ha decidido implementar
distintos módulos, conectados entre ellos, que permiten calcular el estado
del juego. La relación entre los distintos módulos se peude ver en la
figura~\ref{s3:fig:componentes-fpga-a}. Además, para que el juego pueda
funcionar a una velocidad adecuada, pero se pueda comunicar con los
maletines, y además pintar en el monitor vga, cada módulo está controlado
por su propio reloj, como se explica en la sección~\ref{s3:subsubsec:clocking}.
%y se muestra en la figura~\ref{s3:fig:componentes-fpga-clocking}. 


\begin{figure}[h]
  \centering
  \includegraphics[width=1.0\textwidth]{images/fpga_componentes_v2.png}
\caption{Diagrama de componentes implementados en la FPGA. }
\label{s3:fig:componentes-fpga-a}
\end{figure}

Como se muestra en la figura, los módulos de los que está compuesta la
implementación sobre la FPGA son:
\begin{itemize}
\item \emph{Módulo UART:} módulo encargado de leer y enviar información a
  través de la UART de la FPGA. Posee señales de control para indicar
  cuándo un dato se ha recibido, o cuando está ocupada enviando
  información.
\item \emph{Módulo teclado:} traduce la información recibida por la UART a
  comandos internos entendidos por el juego. Este diseño nos permite
  desacoplar el módulo teclado de la UART, y cambiarlo por otros módulos
  que interpreten otros dispositivos de entrada, como puede ser un teclado
  PS2.
\item \emph{Módulo raquetas:} contienen la información de la posición de
  las raquetas. Las teclas leídas del módulo teclado modifican la posición
  de este módulo. Este módulo se encuentra instanciado varias veces, una
  por cada raqueta del juego.
\item \emph{Módulo bola:} calcula en cada pulso del reloj del juego la
  nueva posición de la pelota. Además, recibe la información de las palas
  para calcular los rebotes con ellas, o si un jugador gana el partido
  actual. Si es el caso, el módulo informa al módulo controlador para que
  anote e informe del cambio de puntuación.
\item \emph{Módulo puntuación:} encargado de guardar la información de los
  partidos. Cuando un jugador consigue un tanto, se comunica con el módulo
  de la UART para enviar la información a los maletines.
\item \emph{Módulo VGA:} a partir de la información del módulo bola y de
  los módulos de raquetas, dibuja en un monitor VGA el juego.
\item \emph{Módulo controlador:} encargado de llevar la máquina de estados
  principal del juego y comunicar los diferentes módulos entre sí.
\end{itemize}

\subsubsection{Gestión de los relojes}
\label{s3:subsubsec:clocking}
Aunque el diseño y ejecución del juego sobre la FPGA tiene muchas ventajas
(la mayoría relativas a la facilidad de realizar cálculos en paralelo), es
necesario sincronizar cada uno de los módulos entre ellos. Además, el
diseño elegido para la implementación hace que cada módulo utilice un reloj
distinto. Los distintos relojes utilizados se muestran en la
figura~\ref{s3:fig:componentes-fpga-clocking}, donde cada color refleja el
reloj utilizado en los módulos que se indican.

\begin{figure}[h]
  \centering
  \includegraphics[width=0.8\textwidth]{images/fpga_componentes_timing_v2.png}
  \caption{Distintos relojes utilizados.}
  \label{s3:fig:componentes-fpga-clocking}
\end{figure}


Más concretamente se ha decidido utilizar cuatro relojes sobre la FPGA,
cada uno con un propósito distinto:
\begin{itemize}
\item \emph{Reloj principal (rosa):} reloj principal de la
  placa. Proporciona un pulso cada 100 ns. Es utilizado para leer los datos
  procesados en tiempo real por la UART e informar a las raquetas de su
  nueva posición. Se ha decidido utilizar un reloj rápido ya que se
  consigue reaccionar de manera más rápida a las pulsaciones del
  jugador. Para no perder datos con el módulo de la UART (al tener
  distintos relojes), la información es almacenada en un biestable hasta
  que esta es leída.
\item \emph{Reloj UART (naranja):} se trata del mismo reloj utilizado para
  la transmisión con la UART en los maletines ARM.
\item \emph{Reloj de juego (verde):} utilizado para calcular los
  movimientos de la pelota. Se trata de una división del reloj principal
  del juego, que al ser más lento permite visualizar los movimientos por la
  pantalla y hacer que el jugador sea consciente de lo que ocurre y le dé
  tiempo a reaccionar.
\item \emph{Reloj VGA (azul):} necesario para transmitir por el
  protocolo VGA. Según este protocolo, el reloj debe ir a 25MHz (un cuarto
  del reloj principal del juego).
\end{itemize}

\subsection{Maletines ARM}
\label{s3:subsec:maletines}
% \todo{Escribir esto, haciendo referencia a~\ref{s3:fig:FSM_maletin}. Si se
%   decide mover la imagen a~\ref{s2:subsec:sistema-entero}, inventarse que
%   poner. }\\
% \todo{Sería interesante explicar lo del tecldo, y lo que se  hace para
%   poder conectar varios maletines en serie.}

Como se ha dicho anteriormente, la FPGA utilizada solamente posee una
conexión UART, por lo que el diseño elegido para los maletines es
conectarlos en serie. Además, todos los maletines ejecutan el mismo código,
encargado de transmitir la información de una UART hacia la otra. La
figura~\ref{s3:fig:FSM_maletin} muestra el comportamiento de los maletines.

% Todos los maletines ejecutan el mismo código, como ya se ha dicho
% anteriormente van conectados en serie mandando comandos de maletín a
% la FPGA y recibiendo las puntuaciones de la FPGA a los maletines. \\

% El funcionamiento de los maletines ARM se puede apreciar en la
% figura~\ref{s3:fig:FSM_maletin}. La primera fase es configurar las
% interrupciones de los botones y el teclado matricial, se inicializan
% los display 8 segmentos y se configuran las UARTs. \\


\begin{figure}[h]
  \centering
  \includegraphics[width=0.8\textwidth]{images/maletin_fsm.png}
  \caption{FSM describiendo el comportamiento de los maletines.}
  \label{s3:fig:FSM_maletin}
\end{figure}


Al iniciar las placas, lo primero ejecutado es la inicialización de la
misma, las conexiones UARTs, activación de interrupciones y marcar los
contadores de puntuación a 0. A continuación se entra en un bucle que va a
controlar el juego. Si se recibe una interrupción de un pulsador o del
teclado matricial, esta pulsación se procesa y se envía por la UART1 al
siguiente elemento de la cadena (que puede ser otro maletín o la FPGA). Si
se detecta la interrupción asociada a un puerto UART, pueden distinguirse
dos casos:
\begin{itemize}
\item Interrupción en UART0. Indica que la información recibida proviene de
  otro maletín. En este caso, se procesa la información para indicar que no
  la ha generado el maletín, y se envía por la UART1 al siguiente elemento
  de la cadena (que va a ser la FPGA en el caso de tener únicamente dos
  maletines).
\item Interrupción en UART1. Indica que la información recibida proviene o
  bien de la FPGA, o bien la está retransmitiendo el otro maletín. La
  información es procesada (información relativa a la puntuación), y
  mostrada por los display 8-Segmentos.
\end{itemize}




%
%
%%%
%%% Local Variables:
%%% mode: latex
%%% TeX-master: "../main.tex"
%%% End:


